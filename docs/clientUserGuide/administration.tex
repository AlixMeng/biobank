\chapter{Administration}
\label{chap:administration}

\section{Repository Sites}
\section{Clinics}
\section{Studies}
\section{Container Types}
\section{Containers}
\section{Moving Samples Between Containers}
\section{Sample Types}
\section{Source Vessels}
\section{Shipping Methods}
\section{Activity Status}
\newpage
\section{User Management}

You can open the user/group administration dialog by selecting
\texttt{Administration} $\to$ \texttt{User Management}.
This menu is enabled only if you have been given the ability to manage users.

You will see the dialog shown in \ref{fig:user_management}. It allows you to 
add/delete/edit users, groups, and roles. You will only be able to manage
roles if you are a manager on all sites and centers with permission to edit roles.
You can only edit groups that you are in or that you have complete power over
(if you have the same permissions the group would grant, but through a different
source).

\begin{figure}[H]
  \centering
  \scalebox{0.5}
	   { \includegraphics*{screenshots/administration/user_management_users} }
	   \caption{User management dialog.}
	   \label{fig:user_management}
\end{figure}

\newpage
\subsection{Managing Roles}
A role is a named group of permissions, it makes managing easier by assigning users a
particular role with permissions instead of having to configure every permission for each
group or user.

To add a new role, click the \fbox{+} button of the roles section. The following dialog will
pop up. Enter a unique name and select all the permissions to be a part of that role.
\begin{figure}[H]
  \centering
  \scalebox{0.5}
	   { \includegraphics*{screenshots/administration/edit_role} }
	   \caption{Adding/Editing a role.}
	   \label{fig:add_group}
\end{figure}

\newpage
\subsection{Managing Memberships}
A membership is a group of permissions and roles that apply to the specified Studies
and Centers (i.e. Clinics, Repository Sites, and Research Groups).

Both users and groups have sets of memberships. To add or edit a membership, go to the
"roles and permissions" tab on a user or group and click on the \fbox{+} button or double
click a membership, respectively. Then you will see the following figure:

\begin{figure}[H]
  \centering
  \scalebox{0.5}
	   { \includegraphics*{screenshots/administration/edit_membership} }
	   \caption{Adding/Editing a membership, part 1.}
	   \label{fig:add_group}
\end{figure}

This first section of a membership determines where the membership applies. Select
'All Centers' if the membership applies to all current and future centers, or select
a specfici subset of centers. Do the same for studies, where 'All Studies' applies to all
current and future studies as well.

Once you have chosen where you want this membership to apply, click the \fbox{Next >} button
to see the following page:

\begin{figure}[H]
  \centering
  \scalebox{0.5}
	   { \includegraphics*{screenshots/administration/edit_membership_2} }
	   \caption{Adding/Editing a membership, part 2.}
	   \label{fig:add_group}
\end{figure}

\begin{description}
\item[Grant all permissions and roles] This will grant the user or group every current
and future permission and role on the given centers and studies.
\item[Can create users] Grants the permission to create users, at the given centers
and studies. This option can only be selected if all permissions and roles are granted.
\item[Roles] Select any roles that you wish to have in this membership. The permissions
in the selected roles will automatically be selected in the permissions section and
cannot be unselected unless that role is removed.
\item[Permissions] Add specific permissions not defined in a Role.
\end{description}

\newpage
\subsection{Managing Users}
To add a new user, open the User Management dialog and click on the \fbox{+} button
of the users section. Double click a user to edit them, or right click and select "Edit".
You will see the following dialog window pop up. Enter the
user's details and press the \fbox{OK} button.

The password assigned to the new user is temporary. When the user logs in for
the first time, he/she will be asked to enter the temporary password and
select a new password.

\begin{figure}[H]
  \centering
  \scalebox{0.5}
	   { \includegraphics*{screenshots/administration/edit_user_general} }
	   \caption{Adding a user.}
	   \label{fig:add_user}
\end{figure}

Normally, a user should be assigned to a group, or have memberships added. Otherwise,
they will not be able to sign in to any sites. Note that this is how a user can have all
of their power removed: by removing all their groups and memberships. A user must be
disabled this way if they cannot be deleted because they have already worked on data
and have left comments or made modifications and the user must remain for logging
purposes.

To delete a user, right click the user and select "Delete".

\newpage
\subsection{Managing Groups}
Groups are useful when you want to share the exact same memberships between a set of users.
Configure a group once, then assign many users to that group.

To add a new group, open the User Management dialog and click on the \fbox{+} button
of the groups section. You will see the following dialog window pop up. Enter the group's
details and press the \fbox{OK} button.
\begin{figure}[H]
  \centering
  \scalebox{0.5}
	   { \includegraphics*{screenshots/administration/edit_group_general} }
	   \caption{Adding/Editing a group.}
	   \label{fig:add_group}
\end{figure}

To delete a group, right click the group and select "Delete". To edit the group, double click
the group or right click and select "Edit".

\newpage
\section{Password Change}
A user connected to the application can modify its password. This can be done by switching to the administration view and selecting \texttt{Administration} $\to$ \texttt{Change password}. You will see the following dialog window pop up. Enter the old password and the new password twice and press the \fbox{OK} button. You will then be asked to connect again.
\begin{figure}[H]
  \centering
  \scalebox{0.5}
	   { \includegraphics*{screenshots/administration/change_password} }
	   \caption{Changing the password.}
	   \label{fig:change_password}
\end{figure}

