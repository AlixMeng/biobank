\chapter{Dispatch}
The \emph{Dispatch} feature is used to record the shipping of a set of patient
sample aliquots between two repository sites where all patients and aliquots
belong to a single study.  Aliquots from multiple patients may be included in a
dispatch.

To use the dispatch feature in Biobank the two sites and the study involved
must first be configured (see section \ref{sec:config_send} for instructions).

Prior to sending a dispatch, the administrator of the sending site must create
a \emph{dispatch entry} that records the aliquots that are part of the
shipment. After the entry is created and the dispatch is ready to be sent, the
administrator must mark the entry as Sent. At this point Biobank displays the
dispatch entry as In Transit (see section \ref{sec:dispatch_send}).

Sites that wish to send dispatches can only send aliquots that have been scan
assigned to container positions. Aliquots without a position are not allowed in
a dispatch.

When receiving a dispatch, a technician with the appropriate role must Receive
and Process the dispatch entry in Biobank. Only entries marked as In Transit
can be received and processed. A technician has the option of only marking an
entry as received and then process it at a later time. Processing of a dispatch
entry involves confirming that the aliquots recorded as sent are actually
received. See section \ref{sec:dispatch_receive} for instructions.  The
following errors can happen when sending a dispatch:
\begin{itemize}
  \item The shipment never arrives at the destination site.
  \item One or more aliquots marked as sent are not actually received at the
    receiving site,
  \item One or more aliquots never marked as sent are received at the receiving
    site.
\end{itemize}
Please see sections \ref{sec:dispatch_not_arrive},
\ref{sec:dispatch_missing_aliquots} and \ref{sec:dispatch_extra_aliquots} for
instructions on how to deal with these cases.  If a repository site no longer
sends dispatches, the instructions to remove the dispatch information is given
in section \ref{sec:config_send_remove}.
\section{Configure a Site for Sending Dispatches}
\label{sec:config_send}
\begin{enumerate}
  \item Log into the Biobank Java Client as a user in the Website
    Administrator group.
  \item From the Administration view, edit the site that is to send dispatches.
  \item \label{add_dispatch} Click on Add Dispatch Relation (see Figure
    \ref{fig:dispatch_add_configuration}).
    \begin{figure}[H]
      \centering
      \scalebox{0.5}
      { \includegraphics*{screenshots/dispatch/add_config} }
      \caption{Configuring a site for sending dispatches.}
      \label{fig:dispatch_add_configuration}
    \end{figure}
  \item From the dialog box the pops up, select the study associated with the
    site and the destination site for the shipments (see Figure
    \ref{fig:dispatch_add_config_dest_site}) and press the \emph{OK}
    button. Use the button with an arrow pointing to the right to move a site
    from the \emph{Available Sites} box to the \emph{Selected Destination Sites}
    box.
    \begin{figure}[H]
      \centering
      \scalebox{0.5}
      { \includegraphics*{screenshots/dispatch/add_config_dest_site} }
      \caption{Configuring the destination site for dispatches.}
      \label{fig:dispatch_add_config_dest_site}
    \end{figure}
  \item If there are more studies that dispatch from this site go to step \ref{add_dispatch}.
\end{enumerate}
\section{Remove Dispatch Configuration for a Site}
\label{sec:config_send_remove}
\begin{enumerate}
  \item Log into the Biobank Java Client as a user in the Website
    Administrator group.
  \item From the Administration tree view, right click on the site that will no longer send
    dispatches and select \emph{Edit Site}.
  \item In the entry form that opens up, right click on the study / site you
    wish to remove the configuration for and select \emph{Delete}.
    \begin{figure}[H]
      \centering
      \scalebox{0.5}
      { \includegraphics*{screenshots/dispatch/del_config} }
      \caption{Removing dispatch configuration for a site.}
      \label{fig:dispatch_del_config}
    \end{figure}
\end{enumerate}
\section{Sending a Dispatch}
\label{sec:dispatch_send}
\begin{enumerate}
  \item Log into the Biobank Java Client as a user in the Website
    Administrator group. \todo[size=\small]{which user groups should be allowed to
      create dispatches?}%
  \item In the \emph{Working Site} pull down, select the site that is to send the
    dispatch.
  \item Using the main menu or the toolbar icon select \emph{Dispatch} view.
  \item Select \emph{Add Dispatch} from the \emph{Dispatch Shipments} sub menu
    or from the toolbar icon.
    \begin{figure}[H]
      \centering
      \scalebox{0.5}
      { \includegraphics*{screenshots/dispatch/add} }
      \caption{Creating a new dispatch.}
      \label{fig:dispatch_add}
    \end{figure}
  \item Select the study and receiver site the dispatch aliquots are for. If
    there is a single study and / or a single receiver site they will be
    already selected.
    \begin{figure}[H]
      \centering
      \scalebox{0.5}
      { \includegraphics*{screenshots/dispatch/add_entry} }
      \caption{Selecting study and site when creating a dispatch.}
      \label{fig:dispatch_add_entry}
    \end{figure}
  \item Press the \emph{open scan dialog} button to scan assign aliquots into
    the dispatch and the following dialog box is displayed.
    \begin{figure}[H]
      \centering
      \scalebox{0.35}
      { \includegraphics*{screenshots/dispatch/add_entry_scan} }
      \caption{Scanning aliquots into a dispatch.}
      \label{fig:dispatch_add_entry_scan}
    \end{figure}
  \item Enter the pallet's product barcode or click the \emph{new
    pallet}\footnote{The new pallet check box is used when aliquots are moved
    from their original pallet to a new pallet to be used for the actual shipping.} check
    box, then the plate number for where the pallet has been placed on the
    flatbed scanner and then press the Launch Scan button.
\end{enumerate}
\section{Receiving a Dispatch}
\label{sec:dispatch_receive}
\section{Dispatch Never Arrives at Destination Site}
\label{sec:dispatch_not_arrive}
\section{Dispatches with Missing Aliquots}
\label{sec:dispatch_missing_aliquots}
\section{Dispatch with Extra Aliquots}
\label{sec:dispatch_extra_aliquots}
