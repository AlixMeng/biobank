\documentclass[hyperref={pdfpagelabels=false}]{beamer}
\useoutertheme{infolines}
\usetheme[height=7mm]{Singapore}
\setbeamertemplate{items}[ball]
\setbeamertemplate{blocks}[rounded][shadow=true]
\setbeamertemplate{navigation symbols}{}

\usepackage[english]{babel}
\usepackage[latin1]{inputenc}
\usepackage{times}
\usepackage[T1]{fontenc}

\title[CBSR Overview]{Canadian BioSample Repository}
\subtitle{An Overview}
\author{}
\institute{Prepared by AICML}
\subject{CBSR Overview}

\pgfdeclareimage[height=0.5cm]{logo}{cbsr}
\logo{\pgfuseimage{logo}}


% If you wish to uncover everything in a step-wise fashion, uncomment
% the following command:

%\beamerdefaultoverlayspecification{<+->}


\begin{document}

\begin{frame}
  \titlepage
\end{frame}

%\begin{frame}{Outline}
%  \tableofcontents
%  % You might wish to add the option [pausesections]
%\end{frame}

%\section{Introduction}
%\subsection[Short First Subsection Name]{First Subsection Name}

\begin{frame}{About CBSR}
  \begin{itemize}
    \item Biosample processing, management, storage and retrieval
      infrastructure.
    \item Operates on a cost-recovery basis to investigators in the Canadian
      research community.
    \item Available to Research Ethics Board (REB) approved studies.
    \item Accepts human tissue from appropriately consented subjects for
      processing of DNA, RNA, serum, plasma, peripheral blood cells, fixed
      tissue, and tissue blocks.
    \item Short and long-term sample storage.
    \item Retrieval typically within 24 hours upon request.
  \end{itemize}
\end{frame}

\begin{frame}{About CBSR}{Continued}
  \begin{itemize}
    \item Storage services include robust robotic systems for sample
      aliquoting, tracking, storage, and retrieval
    \item Chain of custody ensured on all samples.
    \item Temperature monitoring systems ensure that all samples are maintained
      at the chosen temperature.
    \item Regulatory-grade audit trail for all samples.
  \end{itemize}
\end{frame}

\begin{frame}{Personnel}
  CBSR currently has the following presonnel:
  \begin{itemize}
    \item Director: Dr. Bruce Ritchie
    \item Financial Supervisor: Meagen LaFave
    \item Technicians: 4 members
    \item Inventory Tracking System software team: 4 members
  \end{itemize}
\end{frame}


\begin{frame}{Technician Primary Tasks}
  The primary tasks performed by the technicians are:
  \begin{itemize}
    \item Sample Processing
    \item Sample Order Processing
    \item Materials Requests
    \item Inventory Audit
    \item Study SOP definition
  \end{itemize}
\end{frame}

\begin{frame}{Sample Processing}{}
  \begin{itemize}
    \item Sample packages received by CBSR from multiple clinics.
      Packages contain biological samples from multiple patients.
    \item The sending clinic follows the collection protocol defined in the
      \emph{Study SOP} (standard operating procedure).
    \item CBSR technician processes the samples according to the SOP. This
      usually involves aliquoting and centrifuging some specimens. Other
      specimens can be stored as is.
    \item Liquid samples are aliquoted into sample tubes. CBSR mainly uses NUNC
      DataMatrix (two dimensional barcode) encoded tubes. Each tube is
      guaranteed to have a unique ID.
    \item Other samples are labeled with an inventory ID encoded into a 1D
      barcode.
  \end{itemize}
\end{frame}

\begin{frame}{Sample Processing}{Continued}
  \begin{itemize}
    \item Technicians use the inventory tracking software to link a sample's
      inventory ID to a patient. Each sample is also linked to a container
      position.
  \end{itemize}
\end{frame}

\begin{frame}{Sample Ordering}
  \begin{itemize}
    \item Research groups that store samples at CBSR can request some or all of
      their samples at any time.
    \item Ordering of samples will soon be automated into the inventory
      tracking system.
    \item Once a research group requests samples, the samples must be shipped
      within 24 hours.
  \end{itemize}
\end{frame}

\begin{frame}{Clinic Materials and Equipment}{}
  CBSR Provides clinics with the following materials and equipment:
  \begin{itemize}
  \item
  \end{itemize}
\end{frame}

\begin{frame}{Storage Details}{}
  \begin{itemize}
  \item Containers used by CBSR are of several types: -80$^\circ$ freezer,
    -180$^\circ$ freezer, room temperature cabinet.
  \item Freezers and cabinets contain subcontainers which themselves can hold
    more subcontainers. Freezers and cabinets each have their own
    subcontainer hierarchy.
  \item Each container and subcontainer in CBSR has a product ID encoded either
    in a 1D or 2D barcode.
  \end{itemize}
\end{frame}

\begin{frame}{Administrative Tasks}
  The primary administration tasks are:
  \begin{itemize}
    \item
  \end{itemize}
\end{frame}

\begin{frame}{Inventory Software}
  \begin{itemize}
  \item The software is a Java based N-tier client server application that
    allows multiple users to access the system at any time.
  \item Based on the National Cancer Intitute's caBIG architecture.
  \item Integrated scanning and decoding software to decode DataMatrix tubes.
  \end{itemize}
\end{frame}

\begin{frame}{Future Plans}{}
  The following projects will soon be undertaken at CBSR:
  \begin{itemize}
  \item Expansion to second site.
  \item Automated sample processing robots
  \item In-freezer robot for tube placement and retrieval
  \end{itemize}
\end{frame}

%% \begin{frame}{}{}
%%   \begin{itemize}
%%     \item
%%   \end{itemize}
%% \end{frame}

\end{document}


