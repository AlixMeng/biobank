\documentclass{beamer}
\mode<presentation>
{
  \usetheme{Copenhagen}
}

\usepackage[english]{babel}
\usepackage[latin1]{inputenc}
\usepackage{times}
\usepackage[T1]{fontenc}

\title[CBSR Overview]{Canadian BioSample Repository}
\subtitle{An Overview}
\author{Prepared by AICML}
\subject{CBSR Overview}

\pgfdeclareimage[height=0.5cm]{logo}{cbsr}
\logo{\pgfuseimage{logo}}


% If you wish to uncover everything in a step-wise fashion, uncomment
% the following command:

%\beamerdefaultoverlayspecification{<+->}


\begin{document}

\begin{frame}
  \titlepage
\end{frame}

%\begin{frame}{Outline}
%  \tableofcontents
%  % You might wish to add the option [pausesections]
%\end{frame}

%\section{Introduction}
%\subsection[Short First Subsection Name]{First Subsection Name}

\begin{frame}{About CBSR}
  \begin{itemize}
    \item Biosample processing, management, storage and retrieval
      infrastructure.
    \item Operates on a cost-recovery basis to investigators in the Canadian
      research community.
    \item Available to Research Ethics Board (REB) approved studies.
    \item Accepts human tissue from appropriately consented subjects for
      processing of DNA, RNA, serum, plasma, peripheral blood cells, fixed
      tissue, and tissue blocks.
    \item Short and long-term sample storage.
    \item Retrieval typically within 24 hours upon request.
  \end{itemize}
\end{frame}

\begin{frame}{About CBSR}{Continued}
  \begin{itemize}
    \item Storage services include robust robotic systems for sample
      aliquoting, tracking, storage, and retrieval
    \item Chain of custody ensured on all samples.
    \item Temperature monitoring system ensures that all samples are maintained
      at the chosen temperature.
    \item Regulatory-grade audit trail for all samples.
  \end{itemize}
\end{frame}

\begin{frame}{Sample Processing}{}
  \begin{itemize}
    \item Sample packages received by CBSR from multiple clinics. Sample
      packages contain multiple biological samples from multiple patients.
    \item The sending clinic follows the collection protocol defined in the
      \emph{Study SOP} (standard operating procedure).
    \item CBSR technician processes the samples according to the SOP. This
      usually involves aliquoting and centrifuging some specimens. Other
      specimens can be stored as is.
    \item Liquid samples are aliquoted into sample tubes. CBSR mainly uses NUNC
      DataMatrix (two dimensional barcode) encoded tubes. Each tube is
      guaranteed to have a unique ID.
    \item Other samples are labeled with an inventory ID encoded into a 1D
      barcode.
  \end{itemize}
\end{frame}

\begin{frame}{Sample Processing}{Continued}
  \begin{itemize}
    \item Technician uses inventory tracking software to link inventory IDs on
      samples to patients. Each samaple is also lined to a container position.
    \item Containers used by CBSR are of several types: -80$^\circ$ freezer,
      -180$^\circ$ freezer, room temperature cabinet.
    \item Freezers and cabinet contain subcontainers, and subcontainers can
      also contain their own subcontainers.
    \item Each container in CBSR has a product ID encoded either in a 1D or 2D
      barcode.
  \end{itemize}
\end{frame}

\begin{frame}{Clinic Materials and Equipment}{}
  \begin{itemize}
    \item CBSR Provides clinics with the following materials and equipment.
      \begin{itemize}
        \item
      \end{itemize}
  \end{itemize}
\end{frame}

\begin{frame}{}{}
  \begin{itemize}
    \item
  \end{itemize}
\end{frame}

\end{document}


