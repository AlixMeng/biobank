\chapter{Configuration}
The configuration options for the Java Client can be found by selecting
\texttt{Configuration -> Preferences} from the main menu.
    \begin{figure}[H]
      \centering
      \scalebox{0.5}
      { \includegraphics*{screenshots/configuration/preferences_dialog} }
      \caption{Configuration preferences dialog box.}
      \label{fig:preferences_dialog}
    \end{figure}
Different configurations pages can be selected by clicking on a node on the
tree on the left hand side of the dialog box. The \texttt{filter text} box can
be used to quickly find a configuration page by typing the name of the page in
whole or in part.

The following sections discuss each configuration page in detail.
\section{Automatic Updates}
\section{General}
\section{Issue Tracker}
\section{Link \ Assign}
\section{Scanning and Decoding}
The items on this preferences page allow the user to specify the scanning and
decoding parameters used to decode 2D barcodes on aliquot tubes.
\begin{center}
\begin{tabular}{|l|p{5in}|}
  \hline
  \textbf{Select Scanner} & Used to select the scanner that will be used to scan
    96 well pallets.\\
  \hline
  \textbf{Driver Type} & The type of driver that was selected when the \emph{Select
    Scanner} button was pressed. Normally, the application will attempt to
    determine the driver type as soon as the user makes the selection, but
    sometimes the application does not select the correct type. Use the
    checkboxes here to override what the application selected.\\
  \hline
  \textbf{DPI} & The \emph{Dots per Inch} to scan images at. For best results use
    600 DPI.\\
  \hline
  \textbf{Brightness} & The brightness setting to be used when scanning
    images. This parameter does not work on Hewlett-Packard scanners when using
    the WIA based driver.\\
  \hline
  \textbf{Contrast} & The contrast setting to be used when scanning
    images. This parameter does not work on Hewlett-Packard scanners when using
    the WIA based driver.\\
  \hline
\end{tabular}
\end{center}
\subsection{Decoding Parameters}

\begin{center}
\begin{tabular}{|l|p{4in}|}
  \hline
  \textbf{Decode Library Debug Level} & The decoding software library can
  output debugging information to a log file. When this value is zero there is
  no debugging information stored in the log file. Possible values are 0
  through 9. The higher the value the more detailed the debugging information.\\
  \hline

  \textbf{Decode Edge Threshold} & Set the minimum edge threshold as a
  percentage of maximum. For example, an edge between a pure white and pure
  black pixel would have an intensity of 100.  Edges with intensities below the
  indicated threshold will be ignored by the decoding process. Lowering the
  threshold will increase the amount of work to be done, but may be necessary
  for low contrast or blurry images. The default and recommended value is
  \emph{5}.\\

  \hline

  \textbf{Decode Square Deviation} & Maximum deviation (in degrees) from
  squareness between adjacent barcode sides. The default and recommended value
  is \emph{N=15} and is meant for scanned images. Barcode regions found with
  corners \emph{<(90-N)} or \emph{>(90+N)} will be ignored by the decoder.\\

  \hline
  \textbf{Decode Corrections} &  The number of corrections to make while
  decoding. The defaut and recommended value is 10.\\
  \hline
  \textbf{Decode Scan Gap} & The scan grid gap size in inches. The
  default and recommended value is \emph{0.085}.\\
  \hline
  \textbf{Decode Cell Distance} & The distance in inches between tubes.The
  default and recommended value is \emph{0.345} for NUNC pallets.\\
  \hline
\end{tabular}
\end{center}

\subsection{Decoding Profiles}
\subsection{Plate Positions}
To define a pallet scanning region do the following:
\begin{enumerate}
  \item On the preferences dialog window, if there is a "plus" symbol next to
    the \emph{Scanning and Decoding} node, press it to expand the sub
    tree.
    \begin{figure}[H]
      \centering
      \scalebox{0.5}
      { \includegraphics*{screenshots/configuration/plate1_definition} }
      \caption{Configuring a plate position.}
      \label{fig:plate1_definition}
    \end{figure}
  \item Place a pallet that contains tubes on the flatbed scanner. Ensure the
    top edge of the pallet is touching the top of the scanning region, and the right
    edge of the pallet is touching the right margin. Ensure the 12 columns
    are vertical and the 8 rows are horizontal.
  \item Select the plate region you are going to define.  If it is the first
    select \emph{Plate 1 Position}.
  \item Click on the "Enable" box.
  \item Press the "Scan" button. The scanner will now scan the entire
    flatbed. You will see something similar to Figure \ref{fig:sample_flatbed_scan}.
    \begin{figure}[H]
      \centering
      \scalebox{0.5}
      { \includegraphics*{screenshots/configuration/sample_flatbed_scan} }
      \caption{Sample flatbed scan.}
      \label{fig:sample_flatbed_scan}
    \end{figure}
  \item Once the scan is done, you are presented with an image that contains
    the entire flatbed and super imposed on it is a grid with 8 rows and 12
    columns. The cell colored in cyan should correspond to tube in row A and
    column 1.
  \item Under orientation select "Landscape".
  \item You can adjust the size of the grid using the mouse. If you move the
    mouse to one of the corners or one of the edges you can resize the grid by
    holding down the left button on the mouse. The whole grid can be moved by
    pressing the left mouse button while hovering inside the grid.
  \item Once the grid cells are aligned with each tube press the "OK"
    button. Use the wheel on the mouse to make the cells smaller or bigger.
  \item Repeat from step 2 to define any more pallet scanning regions.
  \item Usually only one pallet scanning region is required for normal
    operation of the software.
\end{enumerate}
\subsection{Plate Barcodes}
\section{Servers}
